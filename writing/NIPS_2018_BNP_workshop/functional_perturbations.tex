

\paragraph{Sensitivity to functional perturbations.}
%
In order to measure sensitivity to changing the functional form of the prior on
the sticks, we define a parametrized class of multiplicative perturbations to
the base density $p_{0}$ and apply \prettyref{eq:our_approximation}.
Specifically, fix a multiplicative perturbation $\phi(\cdot): [0, 1] \rightarrow
(0, \infty)$ (recall that the stick lengths $\nu_k$  lie in $[0, 1]$). Fix some
$\delta\in[0, 1]$.  We then define a $\delta$-contaminated prior $p_c$ by
%
\begin{align}
\label{eq:expon_perturb}
	p_c(\nu_k \vert \delta, \phi) :=
  \frac{p_{0}(\nu_k)\phi(\nu_k)^\delta}
       {\int_0^1 p_0(\nu_k')\phi(\nu')^\delta d\nu_k'}.
\end{align}
%
The contaminating prior $p_c$ is defined so that $\delta\in[0, 1]$
interpolates multiplicatively between the original prior, $p_{0}$, and a prior
proportional to $\phi(\nu_k)p_{0}$. For example, we might consider a different
prior for the sticks, say $p_1(\nu_k)$.  Letting $\phi(\nu_k) = p_1(\nu_k) /
p_{0}(\nu_k)$, we recover $p_{0}$ at $\delta = 0$ and swap the original
prior for the new prior by taking $\delta \rightarrow 1$.

For a fixed $\phi$, we can use \prettyref{eq:our_approximation} by taking
$\epsilon = \delta$ and
%
\begin{align*}
f^{\delta,\phi}_\eta :=
\frac{\partial^2
    \QExpect \left[ \sum_{k=1}^K \log p_c(\nu_k \vert \delta, \phi) \right]}
{\partial \eta_\theta \partial \delta}
    \Big\rvert_{\eta_\theta = \etathetaopt, \delta = 0} =
\frac{\partial
    \QExpect \left[ \sum_{k=1}^K  \log\phi(\nu_k) \right]}
{\partial \eta_\theta}
    \Big\rvert_{\eta_\theta = \etathetaopt}.
\end{align*}
%
Because we have used a multiplicative perturbation, $f^{\delta, \phi}_\eta$
is linear in $\delta$, which we might expect to improve the fidelity of a
linear approximation.
%
%\footnote{
Indeed, for the purposes of extrapolating to different priors
when using VB based on KL divergence, this
fact appears to recommend multiplicative perturbations amongst the class of
nonlinear perturbations considered by \citet{gustafson:1996:localposterior}.
%}.

% Finally, by linearity of the derivative, the effect of changing all the stick
% priors simultaneously by the same perturbation, $\phi$, is given by taking
% $\delta = \delta_1 = ... = \delta_K$, and
% $\frac{d\etaopt}{d\delta} \Big\rvert_{\eta_\theta = \etathetaopt, \delta=0} =
%     \sum_{k=1}^{K} \frac{d\etaopt}{d\delta_k}
%     \Big\rvert_{\eta_\theta = \etathetaopt, \delta_k=0}$
% \citep{gustafson:1996:localmarginals}.
