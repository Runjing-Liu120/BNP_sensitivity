

\paragraph{Functional perturbations.}
%
In order to measure sensitivity to changing the functional form of the prior on
the sticks, we define a parametrized class of multiplicative perturbations to
the base density $p_{0k}$ and apply \prettyref{eq:our_approximation}.
Specifically, fix a multiplicative perturbation $\phi(\cdot): [0, 1] \rightarrow
(0, \infty)$ (recall that the stick lengths $\nu_k$  lie in $[0, 1]$). Fix some
$\delta\in[0, 1]$.  We then define a $\delta$-contaminated prior $p^c_{k}$ on
the $k$-th stick by
%
\begin{align}
\label{eq:expon_perturb}
	p^c_{k}(\nu_k ; \delta, \phi) :=
  \frac{p_{0k}(\nu_k)\phi(\nu_k)^\delta}
       {\int_0^1 p_0(\nu_k')\phi(\nu_k')^\delta d\nu_k'}
\end{align}
%
$p^c_{k}$ is defined so that $\delta\in[0, 1]$ interpolates multiplicatively
between the original prior, $p_{0k}$, and a prior proportional to
$\phi(\nu_k)p_{0k}$. For example, we might consider a different prior for the
sticks, say $p_1(\nu_k)$; letting $\phi(\nu_k) = p_1(\nu_k) / p_{0k}(\nu_k)$, we
recover $p_{0k}$ at $\delta = 0$ and swap the original prior for the new prior
by taking $\delta \rightarrow 1$.

For a fixed $\phi$, we can use \prettyref{eq:our_approximation} by taking
$\epsilon = \delta$ and
%
\begin{align*}
f^{\delta,\phi}_\eta :=
\frac{\partial^2
    \QExpect \left[ \log p^c_{k}(\nu_k ; \delta, \phi) \right]}
{\partial \eta_\theta \partial \delta}
    \Big\rvert_{\eta_\theta = \etathetaopt, \delta = 0} =
\frac{\partial
    \QExpect \left[ \log\phi(\nu_k) \right]}
{\partial \eta_\theta}
    \Big\rvert_{\eta_\theta = \etathetaopt}
\end{align*}
%
Because we have used a multiplicative perturbation, $f^{\delta,\phi}_\eta$
is linear in $\delta$, which we might expect to improve the fidelity of a
linear approximation
%
\footnote{Indeed, for the purposes of extrapolating to different priors
when using VB based on KL divergence, this
fact appears to recommend multiplicative perturbations amongst the class of
non-linear perturbations considered by \citet{gustafson:1996:localposterior}.}.
% Here,
% %
% \begin{align}
%   \frac{\partial}{\partial \delta} \log p_k^k(\nu_k | \delta, \phi) := \log \phi(\nu_k) -
%     \frac{\int p_0(\nu_k')\log\phi(\nu_k')e^{\delta\phi(\nu_k')} \lambda(d\nu_k')}{\int p_0(\nu_k')\phi(\nu_k')^\delta \lambda(d\nu_k')}
% \end{align}
% %
% The second term does not depend on $\nu_k$, so $f_\eta =
% \frac{\partial}{\partial \eta} E_{q_\eta}[\log \phi(\nu_k)]$.
%
% Now if we perturb each stick $k$ by the same perturbation $\phi(\cdot)$, then
% the derivative of $\eta^*$ with respect to this perturbation is given by
% %
% \begin{align}
%    \frac{d}{d\delta}\eta^*(\delta) &=
%    H^{-1}\frac{\partial}{\partial \eta} E_{q_\eta}\Big[\sum_{k = 1}^K \log \phi(\nu_k)\Big]
%   \label{eq:sensitivity_exp_pert}
% \end{align}
% %
% so derivative for pertubing all the sticks is given by the sum of the deriatives
% of perturbing each stick individually
% \citep{gustafson:2012:localrobustnessbook}.
