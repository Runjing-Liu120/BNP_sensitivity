
We consider the effect of perturbing the prior distribution on the sticks. Let
$p_{0k}(\nu_k; \alpha)$ be the $\text{Beta}(1, \alpha)$ prior on the $k$th
stick, and let $p_{\alpha}(\theta | y)$ be the posterior distribution with a
particular choice of $\alpha$. Now the optimal variational parameters also
depend on $\alpha$, since $\eta^*(\alpha) = \argmin_\eta
KL(q_\eta\left(\theta\right) \| p_{\alpha}(\theta | y))$.

Now suppose we fit the model at some $\alpha_0$, and we wish to re-evaluate the
variational parameters at a different prior parameter $\alpha_1$. Re-optimizing
the objective is computationally expensive, and often, we wish to re-evaluate
the variational parameters at several different prior parameters.

Thus, we propose to approximate the variational parameters $\eta^*$ at
$\alpha_1$ as
%
\begin{align}
    \eta^*(\alpha_1) \approx \eta^*_{lin}(\alpha_1)
    := \eta^*(\alpha_0) +
    \frac{d}{d\alpha}\eta^*(\alpha)\Big|_{\alpha=\alpha_0} \cdot (\alpha_1 - \alpha_0)
    \label{eq:our_approximation}
\end{align}

Following \citet{giordano:2017:covariances},  we compute
$\frac{d}{d\alpha}\eta^*(\alpha) $ as:
%
\begin{align}
  \frac{d}{d\alpha}\eta^*(\alpha)\Big|_{\alpha=\alpha_0} &= H^{-1} f_\eta \label{eq:vb_sensitivty}
\end{align}
%
where
%
\begin{align}
  H &= \frac{\partial^2}{\partial^2\eta}\Big\rvert_{\eta = \eta^*, \alpha = \alpha_0}
  KL(q_\eta\left(\theta\right) \| p_\alpha(\theta | y)) \\
  f_\eta &= \frac{\partial^2}{\partial \alpha \partial \eta}
    \Big\rvert_{\eta = \eta^*, \alpha = \alpha_0} E_{q_{\eta}}
    \big[\log p_\alpha(\theta)\big]
\end{align}

These quantities can be quickly computed with auto-differentiation tools
\citep{maclaurin:2015:autograd}. The derivative also only needs to be computed
once; after computing the derivatve, we can then approximate many values of the
variational parameters under different $\alpha$s.

If $g(\eta)$ is the variational quantity of interest, e.g. the expected number
of posterior quantities in equation \ref{eq:expected_num_clusters}, then we
approximate
%
\begin{align}
    g(\eta^*(\alpha_1)) \approx g(\eta^*_{lin}(\alpha_1))
\end{align}

\subsection{Functional perturbations}
\label{sec:func_pert}
%
We also consider changing the functional form of the prior on the sticks.
Consider some function $\phi(\nu_k)$, where $\phi(\nu_k) > 0$ for all $\nu_k \in
[0, 1]$. Then we define our contaminated prior on stick $k$ as

\begin{align}
	\label{eq:expon_perturb}
	p^c_{k}(\nu_k ; \epsilon, \phi) :=
  \frac{p_{0k}(\nu_k)\phi(\nu_k)^\epsilon}{\int p_0(\nu_k')\phi(\nu_k')^\epsilon \lambda(d\nu_k')}
\end{align}
%
For example, we might consider a different prior for the sticks, say
$p_1(\nu_k)$; letting $\phi(\nu_k) = p_1(\nu_k) / p_{0k}(\nu_k)$, we are
swapping the original prior for a a new prior $p_1$ as $\epsilon \rightarrow 1$.

Here,
%
\begin{align}
  \frac{\partial}{\partial \epsilon} \log p_k^k(\nu_k | \epsilon, \phi) := \log \phi(\nu_k) -
    \frac{\int p_0(\nu_k')\log\phi(\nu_k')e^{\epsilon\phi(\nu_k')} \lambda(d\nu_k')}{\int p_0(\nu_k')\phi(\nu_k')^\epsilon \lambda(d\nu_k')}
\end{align}
%
The second term does not depend on $\nu_k$, so $f_\eta =
\frac{\partial}{\partial \eta} E_{q_\eta}[\log \phi(\nu_k)]$.

Now if we perturb each stick $k$ by the same perturbation $\phi(\cdot)$, then
the derivative of $\eta^*$ with respect to this perturbation is given by
%
\begin{align}
   \frac{d}{d\epsilon}\eta^*(\epsilon) &=
   H^{-1}\frac{\partial}{\partial \eta} E_{q_\eta}\Big[\sum_{k = 1}^K \log \phi(\nu_k)\Big]
  \label{eq:sensitivity_exp_pert}
\end{align}
%
so derivative for pertubing all the sticks is given by the sum of the deriatives
of perturbing each stick individually
\citep{gustafson:2012:localrobustnessbook}.
