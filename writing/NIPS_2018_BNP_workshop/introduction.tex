
A central question in many probabilistic clustering problems is how many
distinct clusters are present in a particular dataset. Bayesian nonparametrics
(BNP) addresses this question by placing a generative process on cluster
assignment, making the number of distinct clusters present amenable to Bayesian
inference.  However, like all Bayesian approaches, BNP requires the
specification of a prior, and this prior may favor a greater or fewer number of
distinct clusters. In practice, it is important to quantitatively establish that
the prior is not too informative, particularly when---as is often the case in
BNP---the particular form of the prior is chosen for mathematical convenience
rather than because of a considered subjective belief.

We derive local sensitivity measures for a truncated variational Bayes (VB)
approximation based on the Kullback-Leibler (KL) divergence. Local sensitivity
measures approximate the nonlinear dependence of a VB optimum on prior
parameters using a local Taylor series approximation
\citep{gustafson:1996:localposterior, giordano:2017:covariances}. Using a
stick-breaking representation of a Dirichlet process, we consider perturbations
both to the scalar concentration parameter and to the functional form of the
stick-breaking distribution. As far as the authors are aware, ours is the first
analysis of the local sensitivity of BNP posteriors when using a VB
approximation.

Unlike previous work on local Bayesian sensitivity for BNP
\citep{Basu:2000:BNP_robustness}, we pay special attention to the ability of our
sensitivity measures to \emph{extrapolate} to different priors, rather than
treating the sensitivity as a measure of robustness \textit{per se}.
Extrapolation motivates the use of multiplicative perturbations to the
functional form of the prior, as the KL divergence is then linear in the
perturbation. Additionally, we linearly approximate only the computationally
intensive part of inference---the optimization of the global parameters---and
retain the non-linearity of easily computed quantities.

We apply our methods to estimate sensitivity to the BNP prior specification of
the expected number of distinct clusters present the Iris dataset
\citep{iris_data_anderson, iris_data_fisher}.  We evaluate the accuracy of our
approximations by comparing to the much more expensive process of re-fitting the
model.
