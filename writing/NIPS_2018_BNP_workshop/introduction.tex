A central question in many probabilistic clustering problems is how many
distinct clusters are present in a particular dataset. In all but the simplest
problems, any attempt to answer this question may be strongly determined by the
criterion used. One common approach, based in the Bayesian tradition, addresses
the problem with a generative model: out of a population of unobserved latent
clusters, some finite number are randomly chosen to be present in the actual
data at hand. The identity and number of these clusters can then be estimated –
with attendant uncertainty estimates – using the tools of Bayesian posterior
inference. For example, one might estimate the “number of distinct clusters
present” by its posterior expectation. Such a generative model is called a
“Bayesian non-parametric” (BNP) model when the number of latent clusters is
infinite, though, naturally, even in non-parametric models only a finite number
of clusters can actually be observed in any particular dataset.

As with any Bayesian model, this approach requires the specification of a prior
and a likelihood. In this case, the likelihood describes the dispersion of data
within a particular cluster, and the prior determines both the distribution of
cluster shapes and sizes as well as the process that determines how many
clusters are present. In general, different choices of the prior and likelihood
would give different answers to the question “how many distinct clusters are
present?” For example, if the prior does not somehow prefer fewer, larger
clusters, then there is nothing that inherently prevents such an approach from
inferring that each datapoint is in its own cluster. However, one still hopes
that a broad range of reasonable choices of prior and likelihood will come to
similar conclusions. Consequently, it is important, in practice, to measure the
sensitivity of the inferred number of clusters present to the prior and
likelihood specification. Furthermore, these sensitivity measures should work
with the kinds of inference tools that are used in practice, operate relatively
automatically without re-fitting the model many times, and measure sensitivity
not only to alternative hyperparameters but also to alternative functional forms
of the prior and likelihood.

To address these needs, we develop fast, automatic measures of the sensitivity
of variational Bayes (VB) approximations to perturbations of functional forms in
a putative model. As a motivating application, we apply our techniques to
estimate the sensitivity of BNP posteriors to the functional form of a
particular BNP prior known as the stick-breaking prior. Stick-breaking priors
provide a strong motivation to quantify functional perturbations. A typical
choice of a stick breaking prior is specified with only a single real-valued
hyperparameter and also a potentially informative distributional assumption, the
form of a stick breaking prior can substantially inform the number of clusters
inferred to be present in a particular dataset, and it is arguably difficult for
ordinary practitioners to form meaningful subjective beliefs about the abstract
form of the stick breaking prior.

We begin by deriving a general result for the sensitivity of VB optima to
function-valued perturbations, as well as several useful specializations. We
then describe a VB approximation to a BNP model with a stick-breaking prior and
derive the sensitivity of the approximate number of inferred clusters to the
choice of the stick breaking prior. We then apply our methods to cluster the
Iris \citep{iris_data_anderson, iris_data_fisher} dataset, comparing our results
to the much more expensive process of re-fitting the model.
