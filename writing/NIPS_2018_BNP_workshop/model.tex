\paragraph{Data and model.}
We use the Iris dataset \citep{iris_data_anderson, iris_data_fisher}, which
contains 150 observations of three different types of iris flowers. We use
measurements of their sepal length, sepal width, petal length, and petal width
to cluster the data with the goal of recovering the three species. Let $y_{n}\in
\mathbb{R}^4$ be these four measurements for flower $n$.

In the spirit of BNP, let us suppose that there there are an infinite number of
distinct species of iris in the world, indexed by $k=1,2,3,\ldots$, only some
finite number of which are present in our observed dataset.  Let $z_n$ denote
the index of the species (i.e. the cluster) to which flower $n$ belongs, i.e.,
$z_n = k$ for exactly one $k$. Each cluster has mean $\mu_k\in
\mathbb{R}^4$ and covariance $\Sigma_k \in \mathbb{R}^{4\times 4}$, and we write
the collections as $\mu = \left(\mu_1, \mu_2, ...\right)$ and $\Sigma =
\left(\Sigma_1, \Sigma_2, ... \right)$. Our data-generating process given the
model parameters is then
%
\begin{align*}
	y_n | z_n, \mu, \Sigma \sim
        \mathcal{N}\left(
            y_n \Big\vert
                \sum_{k=1}^\infty \mathbb{I}\{z_n = k\} \mu_k \;,
              \; \sum_{k=1}^\infty \mathbb{I}\{z_n = k\} \Sigma_k\right),
	\quad n = 1, ..., N.
\end{align*}
%
For $\mu$ and $\Sigma$, we use dispersed IID conjugate priors. For the prior on
the cluster memberships $z_n$, we use a stick-breaking representation of a BNP
Dirichlet process prior \citep{mccloskey:1965:model, ferguson:1973:bayesian,
patil:1977:diversity, sethuraman:1994:constructivedp}. Specifically, we define
latent stick-breaking proportions $\nu=\left(\nu_1, \nu_2, ...\right)$, a concentration
parameter $\alpha>0$, and base stick-breaking distribution $p_{0}\left(\nu_k
\vert \alpha \right) = \mathrm{Beta}\left(\nu_k \Big\vert 1, \alpha \right)$.
The prior on the cluster assignments $z_n$ for $n=1,...,150$ is then given by
%
\begin{align}
\nu \vert \alpha \sim \prod_{k=1}^\infty p_{0}\left(\nu_k \vert \alpha \right)
\textrm{,}
    \quad\textrm{ with }\quad
\pi_k | \nu := \nu_k \prod_{j=1}^{k-1} (1 - \nu_j)
\quad\textrm{ and }\quad
z_n \vert \pi \iid \mathrm{Categorical}(\pi). \label{eq:stick_breaking}
\end{align}
%
The concentration parameter $\alpha$ and stick-breaking prior $p_{0}$
thus determine our prior belief about the number of clusters present. This prior
specification and the observed data combine to inform our posterior belief about the
posterior number of clusters. We will be examining the sensitivity of our
posterior belief to our choices for $\alpha$ and $p_{0}$.
