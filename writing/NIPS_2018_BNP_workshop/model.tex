\paragraph{Data and model.}
We use the Iris dataset \citep{iris_data_anderson, iris_data_fisher}, which
contains 150 observations of three different types of iris flowers. We use
measurements of their sepal length, sepal width, petal length, and petal width
to cluster the data with the goal of recovering the three species. Let $y_{n}\in
\mathbb{R}^4$ be these four measurements for flower $n$.

Let us suppose there are $K$ distinct species of iris in the world, possibly
with $K=\infty$. Let $z_n$ denote the index of the cluster (i.e. the species) to
which flower $n$ belongs, i.e., $z_n = k$ for exactly one $k\in [K]$. Each
cluster has mean $\mu_k\in \mathbb{R}^4$ and covariance $\Sigma_k \in
\mathbb{R}^{4\times 4}$, and we write the collections as $\mu = \left(\mu_1,
\mu_2, ...\right)$ and $\Sigma = \left(\Sigma_1, \Sigma_2, ... \right)$. Our
data-generating process given the model parameters is then
%
\begin{align*}
	y_n | z_n, \mu, \Sigma \sim
        \mathcal{N}\left(
            y_n \Big\vert
                \sum_{k=1}^K \mathbb{I}\{z_n = k\} \mu_k \;,
              \; \sum_{k=1}^K \mathbb{I}\{z_n = k\} \Sigma_k\right),
	\quad n = 1, ..., N.
\end{align*}
%
For $\mu$ and $\Sigma$, we use dispersed IID conjugate priors.
For the prior on the cluster memberships $z_n$, we use a stick breaking
process representation of a BNP Dirichlet process prior
\citep{ferguson:1973:bayesian, sethuraman:1994:constructivedp}. Specifically, we
define latent stick lengths $\nu=\left(\nu_1, \nu_2, ...\right)$, a
concentration parameter $\alpha>0$, and base stick-breaking distribution
$p_{0}\left(\nu_k \vert \alpha \right) = \mathrm{Beta}\left(\nu_k \Big\vert 1,
\alpha \right)$.  The prior on the cluster assignments $z_n$ for
$n=1,...,150$ is then given by
%
\begin{align}
\nu \vert \alpha \sim \prod_{k=1}^K p_{0}\left(\nu_k \vert \alpha \right)
\textrm{,}
% \nu_k \vert \alpha &\iid
%     p_{0k}\left(\nu_k \Big\vert \alpha\right)\textrm{, for }k=1,...,K,
    \quad\textrm{ with }\quad
\pi_k | \nu := \nu_k \prod_{j=1}^{k-1} (1 - \nu_j)
\quad\textrm{ and }\quad
z_n \vert \pi \iid \mathrm{Categorical}(\pi). \label{eq:stick_breaking}
%\textrm{, for }n = 1...,150.
\end{align}
%
The concentration parameter $\alpha$ and stick-breaking prior $p_{0}$
thus determine our prior belief about the number of clusters present.
We will be examining the sensitivity of our posterior beliefs about the
number of clusters present to our choice for $\alpha$ and $p_{0}$.
