%
%
% \section{Calculating the derivative}\label{app:sensitivity}
%
% \citet[Theorem 2]{giordano:2017:covariances} give
% the following expression for
% the derivative $d \etaopt(\epsilon)/ d\epsilon^T \vert_{\epsilon = 0}:$
% %
% \begin{align}
% H &= \frac{\partial^2 KL(\eta, \epsilon)}{\partial \eta \partial \eta^T}
%     \Big\rvert_{\eta = \etaopt, \epsilon = 0} \nonumber \\
% f_\eta &= \frac{\partial^2
%     \Expect_{q\left(\theta, z \vert \eta\right)}
%         \left[ \log p\left(y, \theta, z \vert \epsilon \right)\right]
%     } {\partial \eta \partial \epsilon^T}
%     \Big\rvert_{\eta = \etaopt, \epsilon = 0}  \nonumber  \\
% \frac{d \etaopt(\epsilon)}{d\epsilon^T}\Big|_{\epsilon=0} &=
%     H^{-1} f_\eta. \label{eq:vb_sensitivty}
% \end{align}
% %
% Both $H$ and $f_\eta$ can be evaluated using only $\etaopt(0)$, allowing us to
% use \prettyref{eq:our_approximation} to approximate $\eta$ without re-optimizing
% \prettyref{eq:pert_kl_objective} for different $\epsilon$. Furthermore, $H$ and
% $f_\eta$ can be quickly computed with auto-differentiation tools
% \citep{maclaurin:2015:autograd}. The derivatives and Cholesky factorization of
% $H$ also only need to be computed once, and these calculates can be reused to
% approximate many values of the variational parameters under different
% values of $\epsilon$.
